\section{Stochastic model}
\label{sec:stochastic}
\textbf{move this part (partly) to introduction}The aim of a Stochastic Simulation algorithm (SSA) is to provide a computational model that allows for stochastic reactions. They can be classified as exact and approximate SSAs. Exact algorithms, like the Gillespie-Algorithm \citep{Gillespie.1977}, do not group reactions together but model them one after another. Approximate algorithms, like $\tau-leaping$ \citep{Gillespie.2001}, group reactions together and update the propensities within a larger interval $\tau$. Thus, approximate algorithms might have an advantage when it comes to speed. However, by updating the propensities less often, we lose some accuracy. For speed purposes, we decided to implement the $\tau$-leaping algorithm with efficient step size selection as in \cite{Cao.2006}. \textbf{or stochastic Euler} \\

\textbf{$\tau$-leaping}. In this section we impose an arbitrary order to all subcompartments. We do so by using n features $F_1, \dots F_n$, such that $\cup_{i=1}^n \set(F_i) = \set()$ and $\set(F_1), \dots, \set(F_n)$ are mutually disjoint. We specify the state of the system in terms of the subcompartments $Y(t) = \begin{pmatrix} \num(F_1) & \num(F_2) & \dots & \num(F_n)\end{pmatrix}'$. Let $\vect{S} \in \R^{n \times m}$ be the stoichiometric matrix, as defined in \eqref{eq:ode_system_matrix}, with coefficients $s_{ij}$ and columns $s_{.j}$. Let $R_j$, for $j=1, \dots, m$, be the $j$-th reaction and $V_{t,j}$ the random variable counting the number of times $R_j$ will fire within the interval $[t, \tau)$, for $\tau \in \R_+$. We denote by $V_j$ the random vector collecting the random variables. Dividing the whole period of interest $[0, T]$ in intervals of length $\tau$, the leaping is an iterative update of the discretizised system's state.
\begin{align}
\label{eq:tau_leaping}
Y(t + \tau) =& Y(t) + \vect{S} V_t.
\end{align}
This equation is similar to equation \eqref{eq:sys_change}. Only the number of reactions within a given interval is in \eqref{eq:tau_leaping} random. Let us examine the latter in more detail. The change in the system's state $\Delta Y(t) = Y(t + \tau) - Y(t)=$ can be written in terms of a linear combination of the columns of $\vect{S}$ with random scalars $K_{t,j}$
\begin{align}
\Delta Y(t) = \sum_{j=1}^m K_{t,j} \cdot s_{.j}.
\end{align}
$s_{.j}$ consists of the magnitudes of the mass actions for each compartment $i$ according to reaction $R_j$ and therefore indicates how the state of the system changes if reaction $R_j$ happens. $K_{t,j}$ is the number of occurences of reaction $R_j$. Thus, the product is the system's change due to $R_j$. Aggregating over all reactions yields the total change of the system $R_j$, similar to the deterministic equation \eqref{eq:sys_change}.\\  

So far we have not specified the distribution of $K_t$. We are interested in the conditional joint probability distribution $\prob_t(K_{t,1} = k_{t,1}, \dots, K_{t,m} = k_{t,m}|\tau)$ of the random vector $K_t = \begin{pmatrix}
K_{t,1}, \dots, K_{t,m} \end{pmatrix}'$ conditioned on the state of the system and a fixed interval size. Recall that we have defined $\prob_t$ to be the conditional distribution with respect to the state of the system and we therefore omit to write it explicitly in the condition statement. Assuming independence, we bring the problem down to specifying the margins. Let $a_j(y)$ be the propensity function $\prob_t(K_{t,j}=1|\tau=1)$ of the $j$-th reaction with respect to the state of the system $Y(t)=y$. We assume that for infinitesimal small $dt$
\begin{align}
\prob_t(K_{t,j} = 1|\tau = dt) = a_j(y) \cdot dt,
\end{align}
is the probability that $R_j$ fires once within the interval $[t, t+dt)$ and $\left(K_{t,j}|Y(t), \tau =dt \right)$ is Bernoulli $\mathrm{Ber}(a_j(y) \cdot dt)$ distributed. The Bernoulli assumption is justified by choosing $dt$ infinitesimal small, such that $R_j$ fires at most once almost surely.

If we assume that $a_j(y)$ is constant within $[t, t+\tau)$, we can partition the interval in $\frac{\tau}{dt}$ subintervals with length $dt$. Note that for simplicitly we assume that $\frac{\tau}{dt}$ is an integer. In each of these subintervals $\left(K_{t+s \cdot dt,j}|Y(t), \tau =dt\right)$, for $s=0, 1, \dots, \frac{\tau}{dt} - 1$, is Bernoulli $\mathrm{Ber}(a_j(y) \cdot dt)$ distributed and thus the sum
\begin{align*}
\sum_{s=0}^{\frac{\tau}{dt}-1} \left(K_{t+s \cdot dt, j} | Y(t), \tau = dt\right) \sim \textrm{B}\left(\frac{\tau}{dt}, a_j(y) \cdot dt\right)
\end{align*}
follows a binomial distribution. The practical problem of this Binomial distribution is that sampling from it requires to define a value for $dt$. By definition $dt$ is infinitesimal small, such that we actually aim for $dt \to 0$. The literature has found a solution to circumvent by breaking the problem down to draw from a Poisson random variable that can be specified by $\tau$ and $a_j(y)$.
\begin{theorem}
$\textrm{B}\left(\frac{\tau}{dt}, a_j(y) \cdot dt\right) \xrightarrow{d} \textrm{Po}(a_j(y) \cdot \tau)$ if $dt \to 0$.
\end{theorem}
\begin{proof}
A general Binomial random variable $\textrm{B}(n'
,p')$ converges in distribution to a Poisson random variable $\textrm{Po}(n' \cdot p')$ if $n' \cdot p'$ is constant, $n' \to \infty$ and $p' \to 0$. Note that by definition $\tau$ is fixed and by assumption $a_i(y)$ is constant within $[t, t+\tau)$. Thus, $\lim_{dt \to 0} \frac{\tau}{dt} = \infty$, $\lim_{dt \to 0} a_i(y) \cdot dt = 0$ and $\frac{\tau}{dt} \cdot a_i(y) \cdot dt = \tau \cdot a_i(y)$. Using the convergence property mentioned above yields the result.
\end{proof}
The $\tau$- leaping algorithm we use is the stochastic Euler algorithm
\begin{algorithm}[H]
%\caption{Simulation}
\begin{algorithmic}[1]
%
%\STATE Set $\alpha \in [0,1)$ and $\lambda \in \mathbb{R}_+$
\STATE Initialize $Y(0) = Y_0$ and set fixed $\tau, \vect{S}$
\STATE Initialize $a_i = a_j(Y(0))$ for all $j = 1, \dots, m$ and $t = 0$
\FOR{$\_=1$ to $\frac{maxPeriod}{\tau}$}
\STATE Set y = Y(t)
\STATE Draw $K_{t,j} \sim \text{Po}(a_j(y) \tau ) $ for all $j = 1, \dots, m$
\STATE Compute $Y(t+\tau) = Y(t) + \vect{S} K_t$
\STATE Store $Y(t+\tau)$
\STATE Update $a_j = a_j(Y(t + \tau))$ for all $j = 1, \dots, m$
\STATE Update $t = t + \tau$
\ENDFOR
\end{algorithmic}

\label{alg:seq}
\end{algorithm}

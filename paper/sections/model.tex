\section{Model}
\label{sec:model}
We subdivide the classic SIRD compartments Susceptible ($S$), Infectious ($I$), Recovered ($R$) and Deceased ($D$), into sub-compartments allowing for heterogeneous areas of residence and vaccination states as well as infections with different virus types. Figure \ref{fig:model} illustrates the general model. Individuals either live in country $A$ or country $B$. They are non-vaccinated $U_0$, vaccinated with vaccine one $U_1$ or vaccinated with vaccine two $U_2$. Vaccine $U_1$ represents the messenger ribonucleic acid (mRNA) vaccines and vaccine $U_2$ represents the vector vaccines. We assume that one vaccination shot is sufficient to get the full protection of vaccine $U_1$ or $U_2$. Further, we introduce two virus types. A wild type $V_W$ that serves as baseline variant and a more infectious mutant variant $V_M$.\\
\begin{figure}[h!]
\centering
\includegraphics[scale=0.3]{images/vaccination_pp_blue_orange.png}\\
\begin{flushleft}
\scriptsize{\textit{Note}: Solid lines indicate transition paths and dashed lines indicate infections. Shots below a compartment indicate that individuals from this compartment are vaccinated. Viruses below a compartment indicate that this compartment is infectious. Each compartment is subdivided according to country of residence and vaccination status.}
\end{flushleft}
\caption{Model structure}
\label{fig:model}
\end{figure}

To describe our model, we denote every sub-group of individuals by a set $\set(F_i)$, where $F_i$ is a placeholder for features that the individuals (elements) within the set $\set$ share and $t$ denotes the time at which the set is evaluated. We illustrate the features $F_i$ in the following by providing examples. Let $X_h$ for $h \in \{S, I, R, D \}$ indicate to which general compartment an individual belongs, then, $\set(X_S)$ is the set of all susceptible individuals and $\set(X_I)$ is the set of all infectious individuals at $t$. If we want to distinguish not only between general compartments but additionally between countries of residence, we use the feature $C_j$ for $j \in \{A, B\}$ to indicate that the country of residence is $j$. $\set(X_S, C_A)$ is the set of all susceptible individuals of country A and $\set(X_S, C_B)$ of country B. Analogously, $\set(V_k)$ is the set of all individuals infected with virus $k$ and $\set(U_l)$ is the set of all individuals with vaccination status $l$.\\

We can link sets with the common set operators, e.g. $\set(X_S, C_A) \cup \set(X_S, C_B) = \set(X_S)$ or $\set(X_S) \cap \set(X_I) = \emptyset$. The negation operator $\neg$ is used to indicate that a certain feature applies for all but the specified compartment, e.g. $\set(\neg X_D)$ is the set of all alive individuals. The cardinality $|\cdot|$ represents the respective number of individuals in a set, e.g. $|\set(X_S)|$ equals the number of all susceptible individuals. To shorthand notation, we define $\num(F_i) = |\set(F_i)|$ as the number of individuals within the set $\set(F_i)$. By definition, $\set() = \cup_{h \in \{ S, I, R,D\}} \set(X_h)$ is the set of all individuals. Table \ref{tab:features} gives an overview of all features. \\
\begin{table}[h!]
\centering
\caption{Notation}
\label{tab:features}
\begin{center}
\scalebox{0.8}{
\begin{tabular}{lclp{10cm}}
\hline
\multicolumn{1}{l}{Feature}&\multicolumn{1}{c}{Code}&\multicolumn{1}{c}{Indices}&\multicolumn{1}{c}{Explanaition}\\
\hline
\rule{0pt}{2.6ex}General compartment & $x_i$ &  $i \in \{S, I, R, D\}$ & Individuals can either be Susceptible ($S$), Infectious ($I$), Recovered ($R$) or Deceasesd ($D$). \\
Country of residence & $c_j$ &  $j \in \{A, B\}$ & Individuals can either live in country A or B. We exclude cross-border movements. \\
Virus Type & $v_k$ &  $k \in \{w, m\}$ & An infection can either be caused by the wild type ($w$) or the mutant ($m$) virus. This feature has to be understood as \textit{is} or \textit{has been} infected with type $k$. \\
Vaccine Type & $u_l$ &  $l \in \{0, 1, 2\}$ & An individual can either be vaccinated with vaccine 1 or 2 or being unvaccinated (0). \\
Placeholder & $F_n$ &  $n \in \mathbb{N}$ & A placeholder that is used to address an arbitrary combination of features. $\set(F_n)$ should be read as the set of a fixed but arbitrary compartment. If we need to distinguish between two arbitrary compartments, we use $F_1$ and $F_2$.\\
\hline
\end{tabular}
}
\end{center}
%\begin{tablenotes}
%\scriptsize
%\item Note: 
%\end{tablenotes}
\end{table}

We impose a set of assumptions to the compartments in order to rule out undesired cases within the model.
\begin{assumption}\label{ass:model}
For all $t,r \in \R_+$, $k \in \{W,M\}$ and $w \in [-t, \infty)$ let
\begin{align*}
\tag{\ref{ass:model}.1} 
\label{eq:no_reinfections}
\set(X_I, V_k) \cap \mathcal{C}_{t+r}(X_S) &= \emptyset \\
\tag{\ref{ass:model}.2} 
\label{eq:no_double_vaccinations}
\set(U_1) \cap \mathcal{C}_{t+w}(U_2) &= \emptyset \\
\tag{\ref{ass:model}.3} 
\label{eq:no_cross_border_mobility}
\set(C_A) \cap \mathcal{C}_{t+w}(C_B) &= \emptyset  \\
\set(X_S, V_k) &= \emptyset
\tag{\ref{ass:model}.4}
\label{eq:no_susceptible_infections}
\end{align*}
\end{assumption}
\noindent Assumption \ref{eq:no_reinfections} rules out reinfections such that an individual that has been infected once cannot become reinfected after it had recovered. According to \cite{Roy.2020}, there is evidence that recovered individuals cannot become reinfected but reinfections cannot be ruled out fully. However, the number of reinfected individuals might be negligible, therefore, we do not incorporate reinfections to keep our model parsimonious. Assumption \ref{eq:no_double_vaccinations} implies that an individual only receives one type of vaccine. Receiving one vaccination shot in our model implies that an individual is fully protected according to the vaccine properties, which eliminates the need for a second shot for the same individual. Assumption \ref{eq:no_cross_border_mobility} rules out permanent cross-country movements of individuals. We do so since  permanent movements should not be a main driver of the pandemic but would require to incorporate more compartments. We refrain from incorporating permanent cross-country movements to keep our model parsimonious. We include cross-border infections by classifying a fraction of infections as cross-border infections.
Assumption \ref{eq:no_susceptible_infections} ensures that susceptible individuals cannot be associated with any type of virus. We do so since they have not been infected yet and therefore cannot be classified to one of the two viruses.

\subsection{Deterministic model}
We use a compartment SIRD model to simulate the pandemic, which is based on a system of ordinary differential equations (ODEs). We see every subcompartment as its own chemical species and each transition, e.g. vaccinations, infections, recoveries, and deaths, as a chemical reaction. Thus, our system becomes a chemical reaction network. To make this thesis self-contained, we explain how the dynamics of a chemical reaction network are modeled using ODEs. We can limit ourselves to the case of irreversible reactions since recovered and deceased individuals cannot become infectious again, and infectious individuals recover but cannot become susceptible. \\

Let $\set(F_1), \dots, \set(F_n)$ be $n$ pairwise disjoint sets such that their union comprises all individuals $\cup_{i=1}^n \set(F_i) = \set()$. Every irreversible reaction $R_j$ for $j = 1, \dots, m$, can be expressed as reaction of all compartments
\begin{align}
\underbrace{\nu_{1j} \num(F_1) + \hdots + \nu_{nj} \num(F_n)}_{Reactants} \longrightarrow \underbrace{\mu_{1j} \num(F_1) + \hdots + \mu_{nj} \num(F_n)}_{Products},
\end{align}
where $\nu_{ij} \in \mathbb{N}_0$ and $\mu_{ij} \in \mathbb{N}_0$ are called stoichiometric coefficients. If a compartment $\set(F_i)$ is not a reactant or product within reaction $R_j$, the respective stoichiometric coefficients $\nu_{ij}, \mu_{ij}$ are set to zero. $\nu_{ij}$ describes how much of species $\set(F_i)$ is consumed and $\mu_{ij}$ how much is produced within reaction $R_j$. The difference $\mu_{ij} - \nu_{ij}$ is the total change of $\num(F_i)$ due to one reaction $R_j$.\\

We are not only interested in how one reaction (e.g. one infection, one vaccination, etc.) influences the state of the system but rather how often this happens within an interval $[t, t+\tau]$ for $\tau \in \R_+$. If we restrict us to the case $\tau = 1$, the latter is described according to the law of mass action by
\begin{align}
\label{eq:change_one_unit}
v_{t,j} = r_j  \prod_{i=1}^n \num(F_i)^{\mu_{ij}},
\end{align}
where $r_j$ is a reaction-specific constant. The product is the number of permutations of individuals that can be part of the reaction. It adjusts the frequency of the reactions by the number of individuals of the respective compartments, e.g. more infectious individuals lead to more subsequent infections.

The change in magnitude of $\num(F_i)$ within the interval $[t, t+\tau]$ is given by the sum of the influences of all $m$ reactions
\begin{align}
\label{eq:sys_change}
y_{t+\tau}(F_i) - \num(F_i) = \sum_{j=1}^m (\mu_{ij} - \nu_{ij}) v_{t,j} \tau.
\end{align}
As outlined above, $(\mu_{ij} - \nu_{ij})$ is the stoichiometry that specifies how one reaction influences the system, and $v_{t,j} \tau$ is the number of times reaction $R_j$ happens within the interval $[t, t+\tau]$. Therefore, the product is the influence of $R_j$ on $\num(F_i)$ within $[t, t+\tau]$. Summed over all reactions, this yields the change of $\num(F_i)$. 

We divide both sides of Equation \eqref{eq:sys_change} by $\tau$, let $\tau \to 0$ and plug in Equation \eqref{eq:change_one_unit} to obtain the ordinary differential equation
\begin{align}
\der(F_i) = \sum_{j=1}^m \left[(\mu_{ij} - \nu_{ij}) \underbrace{r_j  \prod_{k=1}^n \num(F_k)^{\mu_{kj}}}_{v_{t,j}} \right].
\end{align}
We write down the equations for all compartments in matrix form to obtain the system of ODEs, which we use subsequently to ease notation
\begin{align}
\label{eq:ode_system_matrix}
\underbrace{\begin{pmatrix}
\der(F_1) \\ \vdots \\ \der(F_n) \end{pmatrix}}_{\dot{Y}(t)} =
\underbrace{\begin{pmatrix}
\mu_{11} - \nu_{11} & \hdots & \mu_{1m} - \nu_{1m} \\
\vdots & \vdots & \vdots \\
\mu_{n1} - \nu_{n1} & \hdots & \mu_{nm} - \nu_{nm} \\
\end{pmatrix}}_{\vect{S}} \cdot
\underbrace{\begin{pmatrix}
v_{t,1} \\ \vdots \\ v_{t,m}
\end{pmatrix}}_{v_t}.
\end{align}
If we additionally fix the initial condition $Y(0) = Y_0$, the problem becomes an initial value problem. 
%For ODE system's it is common to elaborate on the steady-state of the system. Note that \eqref{eq:ode_system_matrix} is a linear mapping in $v$ for which we can compute the kernel $\mathcal{K}=\left\{v \in \R^m | \vect{S} \cdot v = 0 \in \R^n  \right\}$. Each element of the kernel represents a state where $Y(t)=0$ and therefore the system is in a steady- state. If the pandemic reaches a state such that $v \in \mathcal{K}$, we would be stuck there.  \\
We use AMICI \citep{Frohlich.2021} to solve the initial value problem via simulations. AMICI provides the Python interface to interact with the SUNDIALS \citep{Hindmarsh.2005} solvers. We choose the Backward Differentiation Formulas (BDF) methods implemented in SUNDIALS for our implementation  of the initial value problem. The BDF, as all numerical solvers we are aware of, discretizes the ODE system to simulate it. Let $\tau_t \in \R_+$ be a varying step size and $q \in \mathbb{N}$ be the order of the BDF. For coefficients $a_k\in \R$ and $b \in \R_+$, the BDF-q is  
\begin{align}
\label{eq:BDF}
Y(t) = \sum_{k=1}^q a_{t,k} Y(t - \sum_{i=1}^k \tau_{t-i} ) + \tau_t b_t \underbrace{\dot{Y}(t)}_{\vect{S} v_t}.
\end{align}
The function value $Y(t)$ is a linear combination of the last $q$ simulated function values and its derivative. In Equation \eqref{eq:sys_change}, the baseline function value to determine the next value is the previous function value. In Equation \eqref{eq:BDF}, the baseline function value is the weighted sum of the previous $q$ function values. To account for this, the increase of the tangent $\tau_t \dot{Y}(t)$ is corrected by a factor $b_t$ that accounts for the incorporation of the $q$ last function values.

Since $v_t$ depends on $Y_t$ through Equation \eqref{eq:change_one_unit}, Equation \eqref{eq:BDF} defines $Y(t)$ implicitly. Thus, it cannot be simulated directly from Equation \eqref{eq:BDF} and numerical methods must be applied. SUNDIALS offers several nonlinear solver choices, such as Newton iteration, to solve this numerical problem. For more details on the implementation of the BDF and the nonlinear solvers see \cite{Hindmarsh.2019}.
 
\subsection{Stochastic model}
\label{sec:stochastic}
We use the stochastic equivalent of our model to test how the derived strategies perform under uncertainty. \cite{Gillespie.1977} proposed an algorithm modeling the duration between reactions as random variables that are dependent on the system's state. After one reaction fired, the state is updated and the time to the next period is drawn from the random variable with updated probability distribution. Hence, every probability distribution is based on the latest state of the system and the algorithm is therefore classified as \textit{exact}.  
On the contrary, approximate algorithms, like $\tau-leaping$ \citep{Gillespie.2001}, group reactions together and use the number of times a reaction happens within an interval as a random variable. However, the reactions within one interval almost surely do not happen at the exact same time. Therefore, the true probability distributions of the second and subsequent reactions, within the interval, must change with respect to the outcome of the previous reactions. By simulating all reactions at once, approximate algorithms do not account for this update of the probability distributions. 
The drawback of approximation can be justified by a speed-up of approximation methods that are due to fewer function evaluations. Since our model includes a large number of reactions, we decided to use the approximate method and subsequently dive into the math of it \citep{Gillespie.2001}.\\

%Within the deterministic model, we have assumed that for each reaction $R_j$, the number of times the reaction happens within one unit of time is deterministic, see Equation \eqref{eq:change_one_unit}. However, it is more likely the reactions occur randomly over time. To account for this, we test our deterministically derived optimal vaccination strategy and test it in a stochastic set-up, where the number of times reaction $R_j$ happens, is a random variable.  \\

We impose an arbitrary order on all subcompartments. We do so by expressing all compartments by n features $F_1, \dots F_n$ such that $\cup_{i=1}^n \set(F_i) = \set()$ and $\set(F_1), \dots, \set(F_n)$ are mutually disjoint. We specify the state of the system in terms of the compartments $Y(t) = \begin{pmatrix} \num(F_1) & \num(F_2) & \dots & \num(F_n)\end{pmatrix}'$. Recall that $\vect{S} \in \R^{n \times m}$ is the stoichiometric matrix, as defined in Equation \eqref{eq:ode_system_matrix}, with coefficients $s_{ij}$ and columns $s_{.j}$. $R_j$ for $j=1, \dots, m$, is the $j$-th reaction. Let $K_{t,j}$ be the random variable counting the number of times $R_j$ will fire within the interval $[t, t + \tau)$ for $\tau \in \R_+$. We denote by $K_t$ the random vector collecting the random variables $K_{t,1}, \hdots, K_{t,m}$. Dividing the whole period of interest $[0, T]$ in intervals of length $\tau$, the algorithm is an iterative update of the discretized system's state
\begin{align}
\label{eq:tau_leaping}
Y(t + \tau) =& Y(t) + \vect{S} K_t.
\end{align}
Equation \eqref{eq:tau_leaping} is the equivalent of Equation \eqref{eq:sys_change}, with the only difference that the number of reactions within a given interval are random in Equation \eqref{eq:tau_leaping}. Making use of the latter equation, the change in the system's state $\Delta Y(t) = Y(t + \tau) - Y(t)= \vect{S} K_t$ can be expressed in terms of a linear combination of the columns of $\vect{S}$ with random scalars $K_{t,j}$
\begin{align}
\Delta Y(t) = \sum_{j=1}^m K_{t,j} \cdot s_{.j}.
\end{align}
$s_{.j}$ consists of the stoichiometry for each compartment $\set(F_i)$ according to reaction $R_j$ and therefore indicates how the state of the system changes if reaction $R_j$ happens. $K_{t,j}$ is the number of occurrences of reaction $R_j$. Thus, the product is the system's change due to $R_j$. Aggregating over all reactions $R_j$, this yields the total change of the system, similar to the deterministic Equation \eqref{eq:sys_change}.\\  

So far, we have not specified the distribution of $K_t$. We are interested in the conditional joint probability distribution $\prob_t(K_{t,1} = k_{t,1}, \dots, K_{t,m} = k_{t,m}|\tau)$ of the random vector $K_t = \begin{pmatrix}
K_{t,1}, \dots, K_{t,m} \end{pmatrix}'$ conditioned on the state of the system and a fixed interval size $\tau$. We define $\prob_t$ to be the conditional probability with respect to the state of the system and, therefore, we omit to write the condition explicitly within the condition statement. We assume independence of all $K_{t, 1},\hdots, K_{t, m}$ such that the problem simplifies to determine the marginal distributions. Let $a_j(y) = \prob_t(K_{t,j}=1|\tau=1)$ be the propensity function of the $j$-th reaction with respect to the state of the system $Y(t)=y$. We assume that for infinitesimal small $dt$
\begin{align}
\prob_t(K_{t,j} = 1|\tau = dt) = a_j(y) \cdot dt
\end{align}
is the probability that $R_j$ fires once within the interval $[t, t+dt)$ and $\left(K_{t,j}|Y(t), \tau =dt \right)$ is Bernoulli $\mathrm{Ber}(a_j(y) \cdot dt)$ distributed. The Bernoulli assumption is justified by choosing $dt$ infinitesimally small, such that $R_j$ fires at most once almost surely.

For simplicity, we assume that $\frac{\tau}{dt}$ is an integer. If we assume that $a_j(y)$ is constant within $[t, t+\tau)$, we can partition the interval in $\frac{\tau}{dt}$ subintervals with length $dt$. In each of these subintervals the conditional random variable is Bernoulli distributed $\left(K_{t+s \cdot dt,j}|Y(t), \tau =dt\right) \sim \mathrm{Ber}(a_j(y) \cdot dt)$ for $s=0, 1, \dots, \frac{\tau}{dt} - 1$. Thus, the sum
\begin{align}
\sum_{s=0}^{\frac{\tau}{dt}-1} \left(K_{t+s \cdot dt, j} | Y(t), \tau = dt\right) \sim \textrm{B}\left(\frac{\tau}{dt}, a_j(y) \cdot dt\right)
\end{align}
follows a binomial distribution. The practical problem of this Binomial distribution is that sampling from it requires to define a value for $dt$. By definition, $dt$ is infinitesimally small and we aim for $dt \to 0$. Fortunately, $dt \to 0$ leads to a Poisson random variable that can be specified by the known $\tau$ and $a_j(y)$ such that we can sample from it.
\begin{restatable}{theorem}{Poisson}
$\textrm{B}\left(\frac{\tau}{dt}, a_j(y) \cdot dt\right) \xrightarrow{d} \textrm{Po}(a_j(y) \cdot \tau)$ if $dt \to 0$.
\end{restatable}
\begin{proof}
The proof is moved to Appendix \ref{A:convergence_distribution}
\end{proof}
Armed with the probability distribution of the reactions and an update rule for the states in Equation \eqref{eq:tau_leaping}, we can write down the algorithm explicitly. \\
\begin{algorithm}[H]
 \caption{$\tau$-leaping}
 \label{Algo:stochastic}
\SetAlgoLined
\KwResult{$Y(t) \quad \forall t \in [0,T]$}
 Initialize $Y(0) = Y_0, t=0,$ and set fixed $\tau, \vect{S}$\;
 \While{$t < T$}{
  Set y = Y(t)\;
  Update $a_j = a_j(y)$ for all $j = 1,\dots, m$ \;
  Draw $K_{t,j} \sim \text{Po}(a_j \tau ) $ for all $j = 1, \dots, m$ \;
   Compute $Y(t+\tau) = Y(t) + \vect{S} K_t$\;
   Store $Y(t+\tau)$ \;
    
   Update $t = t + \tau$\;
 }

\end{algorithm}


\subsection{Reactions}
From a chemical reaction network's point of view, our model can be divided into three major groups of reactions: 1. Infections, 2. Recoveries and Deaths, and 3. Vaccinations. We group recoveries and deaths since their reactions have the same structure. Subsequently, we state the general and explicit reactions. We define the reactants, products, stoichiometric coefficients, and reaction constants such that the ODE system \eqref{eq:ode_system_matrix} is determined. Further, we emphasize the reaction constants since they incorporate most of the parameterization into our model.


\subsubsection{Infections}
\begin{figure}[h!]
\centering
\includegraphics[scale=0.3]{images/overview_infection.png}\\
\begin{flushleft}
\scriptsize{\textit{Note}: Solid lines indicate transition paths and dashed lines indicate infections. Shots below a compartment indicate that individuals from this compartment can be vaccinated. Viruses below a compartment indicate that this compartment is infectious with the respective virus type. The letters in the top right corner of each compartment indicate the country of residence of individuals within the compartment.}
\end{flushleft}
\caption{Infection structure}
\label{fig:model_infections}
\end{figure}
Figure \ref{fig:model_infections} depicts the structure of the transitions from the susceptible to the infectious compartments. Every infectious individual $i_1 \in \set(x_I)$ can infect a susceptible individual $i_2 \in \set(x_S)$ regardless of their countries of residence. However, we account for the higher chance of becoming infected by an individual from the same country by reducing the influence of the infectious compartments from another country via the reaction constants. \\

In chemical terms, one infectious and one susceptible compartment serve as reactants and yield two infectious compartments that might be subdivided by further features as products
\begin{align}
\label{eq:general_infection}
\num(x_I, F_1) + \num(x_S, F_2) \longrightarrow \num(X_I, F_1) + \num(X_I, F_2).
\end{align}
We use $F_1, F_2$ to indicate that the reactions differ with respect to not explicitly mentioned features, e.g. vaccinated individuals have a lower risk of becoming infected or transitioning the virus, the mutant virus is more infectious, and cross-border infections are scaled by a factor to make them comparatively rare events. If $F_1 \neq F_2$, the stoichiometric coefficients are one. If $F_1=F_2$, they are one for the reactants but two for the product, which is in this case only one compartment. We incorporate the additional features, represented by $F_1$ and $F_2$, into the reaction constants, which we label as \textit{infection constants} and define as
\begin{align*}
\text{infection constant} = \text{infections per day} \times \text{vaccine modifier} \times \text{compartment adjustment}.
\end{align*}
In the following, we elaborate how to define the components of the \textit{infections per day}, the \textit{vaccine modifiers}, and the \textit{compartment adjustments}.\\

To compute the infections per day, we use the average number of contacts between infectious and susceptible individuals and multiply it with the proportion of individuals that become infected while encountering an infectious individual.

Let $c \in R_+$ be the average number of contacts per individual and day and $\alpha \in [0,1]$ be the proportion of susceptible individuals that become infected when encountering a wild type infected individual without both individuals being vaccinated. Let $\eta \in (1, 1/\alpha]$ be the factor with which the mutant is more infectious than the wild type. Then $\beta = \alpha c$ is the average number of individuals infected per day by $i_1$ if $i_1 \in \set(X_I, V_w, U_0)$. If $i_1 \in \set(X_I, V_M, U_0)$, the average infected number increases to $\eta \beta$. We label $\beta$ as \textit{baseline infection constant} since it covers the most basic case where the reaction happens between an unvaccinated susceptible individual and an unvaccinated wild type infected individual. \\

Vaccinations influence infections via two channels. First, vaccinated susceptible individuals are less likely to become infected \citep{Callaway.2021}. Second, vaccinated infectious individuals are less likely to transmit the virus \citep{Harris.2021}.

To account for the influence of the first channel, we introduce the parameters $\delta_{k,l} \in [0,1]$, where $k \in \{W, M\}$ indicates the virus type and $l \in \{ 1,2\}$ the vaccine type. $\delta_{k,l}$ is the reduction in the probability of becoming infected while encountering an infectious individual after being vaccinated. Thus, susceptible individuals are $1 - \delta_{k,l}$ times less likely to become infected while encountering an infectious individual. This is incorporated within the infection constant by multiplying the baseline infection constant with $1 - \delta_{k,l}$ if $i_s \in \set(X_S, U_1) \cup \set(X_I, U_2)$.

We account for the second channel by introducing the parameter $\gamma \in [0,1]$. $\gamma$ is the reduction in the probability of transmitting the virus after being vaccinated, which we assume to be constant over time and across vaccines. Analogously to the first channel, we  multiply the baseline infection constant with $(1 - \gamma)$ if $i_1 \in \set(X_I, U_1) \cup \set(X_I, U_2)$ to reduce the number of infections caused by a vaccinated individual.  \\

So far, we have only defined the average number of contacts per day $c$ of an infectious individual $i_1 \in \set(X_I, F_1)$ but not specified how these contacts are distributed across compartments. We use $\prob_t(i_2 \in \set(X_S, C_j, F_2) | i_1 \in \set(X_I, F_1))$, the conditional probability that the second individual $i_2$ is susceptible and from compartment $\set(X_S, C_j, F_2)$, and multiply it by $\beta$ to get the baseline infection constant adjusted by the average number of contacts between $i_1$ and individuals of the compartment $\set(X_S, F_2)$. As introduced in the previous chapter, we omit to condition on $Y(t)$ explicitly within the probability statement $\prob_t$ and directly define $\prob_t$ to be conditioned on the state of the system $Y(t)$.

We assume that the vaccination status, the type of virus infection, and the exact general compartment ($X_S, X_I, X_R$) of $i_1$ are independent of $\prob_t(i_2 \in \set(X_S, C_j, F_2)) $. Thus, the problem facilitates to find $\prob_t(i_2 \in \set(X_S, F_2) | i_1 \in \set(\neg X_D, C_{j'}))$. 
%This is to express the probability that the individual that can be infected ($i_2$) is susceptible, unvaccinated, and from area two given that individual $i_1$ is infected with the wild type, unvaccinated, and from area one. 
This independence assumption of the vaccination status implies that for a given number of encounters, an unvaccinated individual $i_1$ does not change her contact habits compared to her counterfactual vaccinated self. Note that this does not mean that we assume that vaccinated and unvaccinated individuals have the same average number of contacts, since the probabilities are defined conditionally that an encounter occurs, but rather implies that, proportionally, she does not meet more vaccinated individuals than her unvaccinated counterfactual. Differences in the average number of contacts between vaccinated and unvaccinated individuals can be incorporated implicitly via the vaccination parameter $\delta_{k,l}$. To facilitate notation, we subsequently take the perspective that the infectious individual lives in country A. However, the same math applies to country B. We provide a detailed derivation of the probabilities within Appendix \ref{A:meeting_prob}.
\begin{align}
\label{eq:cond_meeting_prob}
\prob_t(i_2 \in \set(X_S, C_j, F_2)|i_1 \in \set(\neg X_D, C_A)) &= \begin{cases} 
      1-\frac{\num(X_S, C_j, F_2)}{\num(\neg X_D)} \cdot b(d(A, B)), & j = A \\
      \frac{\num(X_S, C_j, F_2)}{\num(\neg X_D)} \cdot b(d(A, B)), & j = B 
   \end{cases}
\end{align}
The probability is essentially the relative population size adjusted by a penalty function $b: \R_+ \to [0,1]$ that depends on the distance between both countries $d(A, B)$ and accounts for fewer cross-border encounters. By mapping the distance into the unit interval, we allow the probability of a cross-border encounter to be at most as high as the relative population size. The distance can be interpreted as geographical distance but it could also serve to incorporate other factors, like favored holiday destinations, that encourage or discourage cross-border encounters. We impose three conditions on the function $b$
\begin{align}
\lim_{d \to \infty} b(d) &= 0 \tag{B.1}\\
b(0) &= 1 \tag{B.2}\\
b(d_1) &< b(d_2) \quad \textrm{if } d_1 > d_2. \tag{B.3}
\end{align}
Condition $(B.1)$ ensures that countries that have a large distance have only small influences on each other. $(B.2)$ defines a rather theoretical case where cross-border encounters are as likely as within-country encounters. $(B.3)$ ensures that countries that have a greater distance have a smaller influence on each other. \\

With the derived specifications of the infections per day, the vaccine modifiers, and the compartment adjustments, we can specify the compartment-specific infection constants. We illustrate this with two examples. 

\begin{example}
Let the reactants be the compartments $\set(X_{I}, C_A, V_W, U_0)$ and $\set(X_{S}, C_B, U_0)$. The corresponding reaction is
\begin{align}
\num(X_{I}, C_A, V_W, U_0) + \num(X_{S}, C_B, U_0) &\xrightarrow{ r_{j_1}} \num(X_{I}, C_A, V_W, U_0) + \num(X_{I}, C_B, V_W, U_0),
\end{align}
where $r_{j_1}$ denotes the infection rate. To account for the infections per day, we use the baseline infection constant $\beta$ since the infectious compartment is infected with the wild type. The susceptible and the infectious compartments are unvaccinated. Therefore, we do not multiply with a vaccine modifier. The compartments are, however, from different countries. We therefore adjust by multiplying with $\frac{\num(X_S, C_B, F_2)}{\num(\neg X_D)} \cdot b(d(A, B))$
\begin{align}
r_{j_1} = \beta \cdot \frac{\num(X_S, C_B, F_2)}{\num(\neg X_D)} \cdot b(d(A, B))
\end{align}
\end{example}

\begin{example}
For the second example, we consider vaccinated and mutant infected compartments $\set(X_{I}, C_A, V_M, U_1)$ and $\set(X_{S}, C_B, U_2)$ as reactants to showcase the influence of the vaccine and the mutant modifier. 
\begin{align}
\num(X_{I}, C_A, V_M, U_1) + \num(X_{S}, C_B, U_2) &\xrightarrow{ r_{j_2}} \num(X_{I}, C_A, V_M, U_1) + \num(X_{I}, C_B, V_M, U_2).
\end{align}

Since the infectious individual is infected with the mutant, we multiply the baseline infection constant with $\eta$. Since the infectious compartment is vaccinated, we multiply the constant with $(1-\gamma)$. Since the susceptible compartment is vaccinated with vaccine $U_2$, we multiply the infection constant with $1-\delta_{M,2}$. The compartments are from different countries. We therefore adjust by multiplying with $\frac{\num(X_S, C_B, F_2)}{\num(\neg X_D)} \cdot b(d(A, B))$
\begin{align}
r_{j_2} = (1-\delta_{M,2}) (1-\gamma) \eta \beta  \frac{\num(X_S, C_B, F_2)}{\num(\neg X_D)} \cdot b(d(A, B)).
\end{align}
\end{example}

To ensure readability, we refrain from writing down all exact infections but they can be derived as shown in the two examples and are depicted in Figure \ref{fig:model_infections}.

\subsubsection{Recoveries and deaths}
\begin{figure}[h!]
\centering
\includegraphics[scale=0.3]{images/overview_recovery.png}\\
\begin{flushleft}
\scriptsize{\textit{Note}: Solid lines indicate transition paths. Shots below a compartment indicate that individuals from this compartment are vaccinated. Viruses below a compartment indicate that this compartment is infectious with the respective virus type. The letters in the top right corner of each compartment indicate the country of residence of individuals within the compartment. The formulas on top of the solid lines indicate the respective reaction constant.}
\end{flushleft}
\caption{Recovery and death structure}
\label{fig:model_recoveries}
\end{figure}
Figure \ref{fig:model_recoveries} depicts the dynamics of the recoveries and deaths. The general reactions are defined by one infectious reactant and one product
\begin{align}
\label{eq:recovery}
    \set(X_I, F_1) & \longrightarrow  \set(X_D, F_1) \\
    \set(X_I, F_1) & \longrightarrow  \set(X_R, F_1). \notag
\end{align}
The change in mass of the reactants (decrease) and the products (increase) is the product of the average number of individuals transitioning out of the infectious compartment $\set(X_I, F_1)$ and the fraction of individuals that transition to the product compartment, either $\set(X_D, F_1)$ or $\set(X_R, F_1)$. \\

Let $\lambda \in \R_+$ be the average number of individuals that transition out of $\set(X_I, F_1)$. We assume that a constant fraction $p \in [0,1]$ of these individuals dies. Hence, $p\lambda$ individuals transition to the deceased and $(1-p)\lambda$ individuals transition to the recovered state. 
The explicit reactions for unvaccinated individuals are for $j \in \{A, B\}$ and $k \in \{W, M\}$
\begin{align}
    \set(X_I, C_j, V_k, U_0) &\xrightarrow{p \lambda} \set(X_D,C_j, V_k, U_0)  \\
    \set(X_I, C_j, V_k, U_0) &\xrightarrow{(1-p) \lambda} \set(X_R,C_j, V_k, U_0) \notag
\end{align}
$1/\lambda$ is the average duration an individual spends within $\set(X_I, F_1)$. We have implicitly assumed that this time is the same for dying and recovering individuals, which might not be accurate in real-world examples since dying individuals have more severe cases and heavier viral loads, such that they stay infectious longer. However, incorporating separated average durations would raise the need for more compartments. Since we allow for vaccinations of recovered and infectious individuals, slower transitions to the recovered compartments do not bias the vaccine allocation. Therefore, we assume that this simplification is negligible. Moreover, note that $p$ does not depend on the virus type. The virus type therefore only influences the number of infections but not the probability of dying for infected individuals. According to \cite{Davies.2021}, this assumption might be violated due to higher mortality of mutants. However, we do so since the difference is rather small and we want to keep our model simple. \\


If the infectious individuals are vaccinated, they are less likely to decease \citep{Tenforde.2021, Voysey.2021}. To account for this reduction in the fraction that transitions to the deceased state, we introduce the parameters $\omega_{k,l} \in [0,1]$ for $k \in \{W, M\}$ and $l \in \{1,2\}$. We use $p \omega_{k,l}$ as a new probability of dying due to being infected with virus $k$ after being vaccinated with vaccine $l$. $\omega_{k,l}$ is thus the reduction in the probability of dying. The corresponding reactions for vaccinated individuals are for $j \in \{A, B\}, k \in \{W, M\}$ and $l \in \{1,2\}$
\begin{align}
    \set(X_I, C_j, V_k, U_l) &\xrightarrow{p \omega_{k,l} \lambda} \set(X_D,C_j, V_k, U_l) \\
    \set(X_I, C_j, V_k, U_l) &\xrightarrow{(1-p \omega_{k,l}) \lambda} \set(X_R,C_j, V_k, U_l) \notag.
\end{align}

\subsubsection{Vaccination}
\begin{figure}[h!]
\centering
\includegraphics[scale=0.3]{images/overview_vaccination.png}\\
\begin{flushleft}
\scriptsize{\textit{Note}: Solid lines indicate transition paths. Shots below a compartment indicate that individuals from this compartment are vaccinated with the respective vaccine. Viruses below a compartment indicate that this compartment is infectious. The letters in the top right corner of each compartment indicate the country of residence of individuals within the compartment. The formulas next to the solid lines indicate the respective reaction constant.}
\end{flushleft}
\caption{Vaccination structure}
\label{fig:model_vaccination_ov}
\end{figure}
Figure \ref{fig:model_vaccination_ov} depicts the vaccination dynamics. We allow for vaccinations of susceptible, recovered, and deceased individuals. We vaccinate susceptible individuals to protect them from becoming infected. We vaccinate infectious individuals to account for asymptomatic cases \citep{Byambasuren.2020} that receive the vaccine in the real world. Adding asymptomatic infectious compartments would make the model more realistic but increases the number of compartments by 36. We therefore decided to refrain from incorporating it but remark that this could be subject to further research. We vaccinate recovered individuals to consider the vaccine doses taken by them. The latter are vaccinated within the real world to increase their immunity \citep{Skelly.2021}. Since we use real-world vaccine inflow, we found it plausible to incorporate vaccinations of recovered individuals into our model.\\

Let $\phi_{l,j} \in \R_+$ be the vaccination constant of vaccine $l$ in country $j$ at time $t$. The vaccination constant of a vaccine is assumed to be equal for all vaccination subcompartments of $S, I,R$ within one country, indicating that the decision to vaccinate an individual is independent of whether the individual is susceptible, infectious, or recovered.  The corresponding reactions are for $j \in \{A,B\}$, $k \in \{W,M\}$ and $l \in \{1,2\}$
\begin{align}
\set(X_S, C_j, U_0) &\xrightarrow{\phi_{l, j}} \set(X_S, C_j, U_l)  \\
\set(X_I, C_j, V_k, U_0) &\xrightarrow{\phi_{l,j}} \set(X_I, C_j, V_k, U_l) \notag \\
\set(X_R, C_j, V_k, U_0) &\xrightarrow{\phi_{l,j}} \set(X_R, C_j, V_k, U_l). \notag
\end{align}
The vaccination constant is determined by the implemented vaccination strategy. How the vaccination constant is derived given a vaccination strategy, we explain in Chapter \ref{sec:vaccine_allocation}.



\pagenumbering{arabic}
\section{Introduction}
The first case of the COVID-19 pandemic was registered in December 2019 in Wuhan, China \citep{Hui.2020}. Up to July 2021, the pandemic has globally caused around 187 million reported infections following around 4 million deaths \citep{ecdc.2021a}. Actual infection counts are estimated to be even higher due to asymptomatic unregistered cases \citep{Byambasuren.2020}. Policymakers have responded to the outbreak of the pandemic with non-pharmaceutical interventions, such as minimizing the contact numbers of individuals in order to reduce the spread of the virus \citep{Gabler.2021}.  Economists have tried to quantify the costs of these containment measures. \cite{Miles.2020} have estimated the costs of the lockdown in the United Kingdom from March 2020 to beginning of June 2020 to be between 100 and 200 billion pounds, which is 5-10\% of the United Kingdom's GDP. \cite{Deb.2020} have used world-wide data and estimated a 15\% loss in industrial production up to 30 days after the implementation of containment measures. \\

Since the outbreak, more infectious virus mutants have spread, further increasing the physical and social costs. The European Centre for Disease Prevention and Control (ECDC) classifies the virus types into three categories. For (1) \textit{variants of concern}, there is already clear evidence of a significant impact on infections or severity of the disease. For (2) \textit{variants of interest}, this evidence is still preliminary, and (3) \textit{variants under monitoring} have been detected to potentially have the aforementioned impacts \citep{ecdc.2021b}. \\ 


Despite the fact that non-pharmaceutical measures and tests have yielded desired reductions in infections \citep{Gabler.2021}, the high physical and economical costs have raised the need for vaccinations that provide immunity.
In September 2020, more than 100 vaccines against COVID-19 were in the development phase \citep{Mullard.2020}. In late 2020, the first vaccines have been approved. As of July 2021, four vaccines are approved within the European Union (EU) and two are in the development phase \citep{ECa.2021}. In mid July 2021, around 65\% of adults within the European Economic Area have at least received one shot and about half of the adult population have received full vaccination \citep{ecdc.2021a}. With the agreement of all member states, the European Commission have taken on negotiations with the vaccine suppliers representing the member states. Countries indicate during the negotiation phase with a manufacturer if they are interested in the respective vaccine. Vaccine doses are subsequently allocated between the interested member states according to their relative population size \citep{ec.2021}. \\

Our research addresses the question whether the European Union could allocate the purchased vaccines more efficiently through more flexible vaccination strategies than constant rates. Attempting to answer this question, we develop a deterministic Susceptible, Infectious, Recovered, Deceased (SIRD) model with two countries, two vaccines, and two virus types and calibrate it using parameters from the literature. We utilize the real-world vaccine purchase numbers of the EU as exogenous vaccine inflow and scale it down to the population size of our two-country model. Vaccines in our model have varying efficacies with respect to the variants. Variants are distributed heterogeneously across countries. We use piecewise constant functions and logistically transformed cubic Hermite splines as vaccination channels that determine the fractions of the vaccine doses each country receives. Both channels allow the fraction to be non-constant over the time course of the pandemic yielding potentially more complex vaccination strategies. We minimize the number of deaths by optimizing over the parameters of the channel functions, once with additional Pareto constraints and once without the additional constraints. Subsequently, we benchmark the results against the current population size based strategy. We further validate our deterministically derived optimal strategies within a stochastic model.\\


Closest to our research is the work of \cite{Bertsimas.2020}. They use a one-country DELPHI model, an extension to the SEIR model, with different regions of the United States to quantify the effects of vaccinations and demographics within one country. \cite{Matrajt.2021} examine the optimal vaccination strategy with respect to the prioritization of groups within one country. \cite{Tuite.2021} aim to find an optimal policy that allocates vaccines efficiently among first and second vaccine shots.


Our work extends existing research by exploring whether knowledge about the spread of virus types and the respective vaccine efficacies against different types can be used to minimize the number of deceased individuals. We place focus on the importance of policy-wise implementable strategies, such as Pareto improvements with respect to previously applied strategies. \\

We structure the paper as follows. In Section 2, we introduce our deterministic and stochastic SIRD models, review how the system of differential equations can be derived from modeling the transitions as chemical reactions, and specify the corresponding reactions. In Section 3, we introduce the two vaccination channels and specify the optimization problem for the deterministic model. Section 4 elaborates on the calibration of our model and presents the results. Section 5 concludes. 


%\begin{itemize}
%\item thesis serves for both: epidemiologists inetrested in vaccine distribution but also for economcists or social scientists interested in compartment modeling. 
%\end{itemize}
\subsection{Stochastic model}
\label{sec:stochastic}
Within the deterministic model, we have assumed that for each reaction $R_j$, the number of times the reaction happens within one unit of time is deterministic, see Equation \eqref{eq:change_one_unit}. However, it is more likely the reactions occur random over time. To account for this, we test our deterministically derived optimal vaccination strategy and test it in a stochastic set-up, where the number of times reaction $R_j$ happens, is a random variable.  \\

\textbf{$\tau$-leaping}. We impose an arbitrary order to all subcompartments. We do so by using n features $F_1, \dots F_n$, such that $\cup_{i=1}^n \set(F_i) = \set()$ and $\set(F_1), \dots, \set(F_n)$ are mutually disjoint. We specify the state of the system in terms of the subcompartments $Y(t) = \begin{pmatrix} \num(F_1) & \num(F_2) & \dots & \num(F_n)\end{pmatrix}'$. Recall that $\vect{S} \in \R^{n \times m}$ is the stoichiometric matrix, as defined in \eqref{eq:ode_system_matrix}, with coefficients $s_{ij}$ and columns $s_{.j}$. $R_j$, for $j=1, \dots, m$, is the $j$-th reaction. Let $V_{t,j}$ be the random variable counting the number of times $R_j$ will fire within the interval $[t, t + \tau)$, for $\tau \in \R_+$. We denote by $V_t$ the random vector collecting the random variables $V_{t,1}, \hdots, V_{t,m}$. Dividing the whole period of interest $[0, T]$ in intervals of length $\tau$, the leaping is an iterative update of the discretizised system's state
\begin{align}
\label{eq:tau_leaping}
Y(t + \tau) =& Y(t) + \vect{S} V_t.
\end{align}
This equation is the quivalent to equation \eqref{eq:sys_change}, with the only difference that the number of reactions within a given interval is random in Equation \eqref{eq:tau_leaping}. \\

The change in the system's state $\Delta Y(t) = Y(t + \tau) - Y(t)= \vect{S} V_t$ can be written in terms of a linear combination of the columns of $\vect{S}$ with random scalars $K_{t,j}$
\begin{align}
\Delta Y(t) = \sum_{j=1}^m K_{t,j} \cdot s_{.j}.
\end{align}
$s_{.j}$ consists of the stoichiometry for each compartment $i$ according to reaction $R_j$ and therefore indicates how the state of the system changes if reaction $R_j$ happens. $K_{t,j}$ is the number of occurences of reaction $R_j$. Thus, the product is the system's change due to $R_j$. Aggregating over all reactions yields the total change of the system $R_j$, similar to the deterministic equation \eqref{eq:sys_change}.\\  

So far we have not specified the distribution of $K_t$. We are interested in the conditional joint probability distribution $\prob_t(K_{t,1} = k_{t,1}, \dots, K_{t,m} = k_{t,m}|\tau)$ of the random vector $K_t = \begin{pmatrix}
K_{t,1}, \dots, K_{t,m} \end{pmatrix}'$ conditioned on the state of the system and a fixed interval size. Recall that we have defined $\prob_t$ to be the conditional probability with respect to the state of the system and we therefore omit to write the condition explicitly within as condition statement. Assuming independence of all $K_{t, 1},\hdots, K_{t, m}$, we simplify the problem to specifying the marginal distributions. Let $a_j(y)$ be the propensity function $\prob_t(K_{t,j}=1|\tau=1)$ of the $j$-th reaction with respect to the state of the system $Y(t)=y$. We assume that for infinitesimal small $dt$
\begin{align}
\prob_t(K_{t,j} = 1|\tau = dt) = a_j(y) \cdot dt,
\end{align}
is the probability that $R_j$ fires once within the interval $[t, t+dt)$ and $\left(K_{t,j}|Y(t), \tau =dt \right)$ is Bernoulli $\mathrm{Ber}(a_j(y) \cdot dt)$ distributed. The Bernoulli assumption is justified by choosing $dt$ infinitesimal small, such that $R_j$ fires at most once almost surely.

For simplicitly we assume that $\frac{\tau}{dt}$ is an integer. If we assume that $a_j(y)$ is constant within $[t, t+\tau)$, we can partition the interval in $\frac{\tau}{dt}$ subintervals with length $dt$. In each of these subintervals the conditional random variable is Bernoulli distributed $\left(K_{t+s \cdot dt,j}|Y(t), \tau =dt\right) \sim \mathrm{Ber}(a_j(y) \cdot dt)$ for $s=0, 1, \dots, \frac{\tau}{dt} - 1$. Thus, the sum
\begin{align}
\sum_{s=0}^{\frac{\tau}{dt}-1} \left(K_{t+s \cdot dt, j} | Y(t), \tau = dt\right) \sim \textrm{B}\left(\frac{\tau}{dt}, a_j(y) \cdot dt\right)
\end{align}
follows a binomial distribution. The practical problem of this Binomial distribution is that sampling from it requires to define a value for $dt$. By definition $dt$ is infinitesimal small, such that we actually aim for $dt \to 0$. Fortunately, $dt \to 0$ leads to a Poisson random variable that can be specified by the known $\tau$ and $a_j(y)$.
\begin{theorem}
$\textrm{B}\left(\frac{\tau}{dt}, a_j(y) \cdot dt\right) \xrightarrow{d} \textrm{Po}(a_j(y) \cdot \tau)$ if $dt \to 0$.
\end{theorem}
\begin{proof}
The proof is moved to the Appendix \ref{A:convergence_distribution}
\end{proof}

The $\tau$- leaping algorithm we use is the stochastic Euler algorithm with fixed step-size. We use a fixed step-size since we have a predefined step-size within the deterministic model and to test the deterministically derived vaccination strategies properly, we decided to use the same step-size for the stochastic model. The algorithm requires the initialization of the system $Y(0)$ and specifying the stoichiometry. Within the update steps, it mainly exploits Equation \eqref{eq:tau_leaping}.\\

\begin{algorithm}[H]
 \caption{Stochastic Euler algorithm}
\SetAlgoLined
\KwResult{$Y(t) \quad \forall t \in [0,T]$}
 Initialize $Y(0) = Y_0, t=0,$ and set fixed $\tau, \vect{S}$\;
 \While{$t < T$}{
  Set y = Y(t)\;
  Update $a_j = a_j(y)$ for all $j = 1,\dots, m$ \;
  Draw $K_{t,j} \sim \text{Po}(a_j \tau ) $ for all $j = 1, \dots, m$ \;
   Compute $Y(t+\tau) = Y(t) + \vect{S} K_t$\;
   Store $Y(t+\tau)$ \;
    
   Update $t = t + \tau$\;
 }

\end{algorithm}

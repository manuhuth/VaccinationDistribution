\documentclass[12pt, american]{article}%
\usepackage{amssymb}
\usepackage{amsfonts}
\usepackage{float}
\usepackage{amsmath}
\mathcode`*=\string"8000
\begingroup %makes * prettier
\catcode`*=\active
\xdef*{\noexpand\textup{\string*}}
\endgroup
\usepackage{lmodern}
\usepackage{tikz}
\usepackage[nohead]{geometry}
\usepackage[doublespacing]{setspace}
\usepackage[bottom]{footmisc}
\usepackage{booktabs} % optisch bessere Tabellen
\usepackage{endnotes}
\usepackage{natbib}
\usepackage{comment}
\usepackage{graphicx,caption}% ,subfigure} use subcaption instead, see below
\usepackage{subcaption}
\usepackage{rotating}
\usepackage{hyperref}
\usepackage[latin1]{inputenc} % Zeichencodierung
\usepackage[T1]{fontenc} % Zeichencodierung
\usepackage{pdflscape}
\usepackage{rotating}
\usepackage[flushleft]{threeparttable}
\usepackage{multirow}
\usepackage{dsfont}
\usepackage{dcolumn}% An Tabellen sollten die Spalten am Dezimalpunkt ausgerichtet sein! Dafür statt 'c' etc. einfach einen Punkt '.' setzen! Dies wird % durch nachstehenden Befehl definiert.
\newcolumntype{.}{D{.}{.}{4}}
\usepackage{%
  babel,     % Babel fuer diverse Sprachanpassungen
  fixltx2e,  % Verbessert einige Kernkompetenzen von LaTeX2e
}
\setcounter{MaxMatrixCols}{30}
\newenvironment{proof}[1][Proof]{\noindent\textbf{#1.} }{\ \rule{0.5em}{0.5em}}
\makeatletter
\def\@biblabel#1{\hspace*{-\labelsep}}
\makeatother
\geometry{left=2.6cm,right=2.6cm,top=2.6cm,bottom=2.6cm}
\hypersetup{colorlinks, linkcolor=blue, citecolor=blue, urlcolor=blue}
%\usepackage{cite} % ohne eckige Klammer in Text
%\usepackage[noadjust]{cite} % ohne eckige Klammer in Text
% \renewcommand\citeleft{}%
% \renewcommand\citeright{}%
% % %
\renewcommand*{\sectionautorefname}{Section}
\renewcommand*{\subsectionautorefname}{Subsection}
%\renewcommand*{\equationautorefname}{}  % eliminate equation in autoref

\newcommand{\vect}[1]{\mathbf{#1}}
\newcommand{\thin}{\thinspace}
\newcommand{\thick}{\thickspace}
\newcommand{\N}{\mathcal{N}}	%Normal Distribution
\newcommand{\U}{\mathrm{U}}	%Uniform Distribution
\newcommand{\D}{\mathrm{D}}	%Dirichlet Distribution
\newcommand{\W}{\mathrm{W}}	%Wishart Distribution
\newcommand{\E}{\mathbb{E}}	%Expectation
\newcommand{\prob}{\mathbb{P}} %probability
\newcommand{\R}{\mathbb{R}} %real numbers
\newcommand{\Z}{\mathbb{Z}} %integers
\newcommand{\Ind}{\mathbb{I}\,}	%Indicator Function
\newcommand{\F}{\mathcal{F}} %history
\newcommand{\ic}{\text{i}} %complex number
\newcommand{\e}{\text{e}} %eularian number
\newcommand{\diag}{\text{diag}} %eularian number
\newcommand{\M}{\mathcal{M}}
\newcommand{\x}{\mathbf{x}}
\newcommand{\X}{\mathbf{X}}
\newcommand{\set}{\mathcal{P}}	

%----------------------------------------------------
\DeclareMathOperator*{\argmin}{arg\,min}
\DeclareMathOperator*{\argmax}{arg\,max}

\newcommand{\TR}{\mathrm{TR}\,}
\newcommand{\OV}{\mathrm{OV}\,}

\newcommand{\bs}{\boldsymbol}
\newcommand{\var}{\mathbb{V}\thin}
\newcommand{\vecf}{\mathrm{vec}\thin}
\newcommand{\plim}{\overset{p}{\to}}
\newcommand{\cov}{\mathrm{Cov}\thin}
\newcommand\indep{\protect\mathpalette{\protect\independenT}{\perp}}
\def\independenT#1#2{\mathrel{\rlap{$#1#2$}\mkern5mu{#1#2}}}
\usepackage{bbm}
%\usepackage{endfloat}
\renewcommand{\vec}[1]{\mathbf{#1}}
\DeclareMathOperator{\tr}{tr}

\let\proof\relax
\let\endproof\relax
\usepackage{amsthm}
\theoremstyle{plain}
\newtheorem{assumption}{Assumption}
\theoremstyle{plain}
\newtheorem{theorem}{Theorem}
\theoremstyle{definition}
\newtheorem{proofof}{Proof of Theorem}
\theoremstyle{plain}
\newtheorem{algorithm}{Algorithm}

\interfootnotelinepenalty=10000


\definecolor{pyBlue}{RGB}{31, 119, 180}
\definecolor{pyOrange}{RGB}{255, 127, 14}
\definecolor{pyGreen}{RGB}{44, 160, 44}
\section{Conclusion}
Our research aims to trace out if the population size based EU vaccination policy against the COVID-19 pandemic can be improved, such that it minimizes deaths across countries. We propose a deterministic SIRD compartment model, with two countries and two vaccines, to examine the effect of optimal strategys compared to the current EU strategy. We calibrate our model with parameters from the literature and the real-world vaccine inflow of the COVID-19 vaccines. We construct the vaccination channels via piecewise constant functions and splines. We examine one case where we impose further restrictions to the optimal vaccine allocation and one case where constrain the results to be a an improvement for each country. We simulate the models numerically and validate the optimal strategies derived from the deterministic model in a stochastic extension to our model.\\

Our results show that the optimal derived vaccination strategies differ from the current EU strategy, which could indicate that more complex vaccination policies are able to lower the number of death cases caused by the pandemic. Enhancing the robustness of our results, we find qualitatively highly similar results using piecewise constant and spline vaccine channels. Leaving country specific interests aside, we find it can be beneficial to assign most of the vaccine doses to one country, leading to a remarkable reduction of deaths within this country. The other country experiences a severe increase in death cases but the decrease of case numbers within the first country is substantially lower, leading to an overall decrease in the number of deaths. However, policy makers of the second country would not be willing to agree to this policy due to the higher case numbers.  We find that imposing additional Pareto constraints yields an overall improvement in comparison to the current strategy but an overall deterioration in comparison to the unrestricted optimal strategy. The Pareto optimal strategy, we derive, assigns just enough vaccine doses to the second country such that it has the same case numbers but the first country is better off, leading to an overall improvement to the current strategy. Since both countries are not worse off, they have no incentive to vote against the Pareto optimal strategy, making it more likely to be implemented in practice. \textbf{one sentence about stochastic validation}\\

In ongoing work, we pursue two avenues for further improvement. First, we implement neural networks as third channel of vaccine allocation. The piecewise constant approach and the splines are not constructive since we do not link the vaccine allocation directly to the state of the model. For the neural network, we use model states and parameters that could be (at least approximately) observed in the real-world, such as the current infected case numbers, the change in infected case numbers, and the reproduction number, as inputs. An optimal allocation is in this scenario reproducible within the real world in a sense that inputs can be observed and the real-world vaccination strategy can be adjusted according to the neural network. Second, we plan to incorporate demographics in our model by adding further compartments that account for different age structures of countries. Adding further age dependent compartments helps us to understand how vaccination strategies must be adapted according to demographics. We especially aim to focus on the implications of vaccinating children.  \\

%Third, we aim to calibrate our model with data to derive the parameters empirically and to validate our simulations. In addition, we implement several 
%sensitivity checks, such as changing the sensitivity checks for several parameters: hyperbolic tangent instead of logistic function for splines, parameter values, distance!, multi-country example
%\textbf{What is currently in planning}
%\begin{itemize}

%\item stochastic optimization
%\end{itemize}


In addition, it might be worth to examine how the dynamics of the model change by including tests and quarantines. We are particularly interested in examining how testing and vaccinations can be optimally  used in composition, or if tests can even substitute vaccinations up to a certain degree. Moreover, our analysis does not take non-death related disutilities, like physical long-term damage caused by an infection into account. A modified objective that takes long-term measures into account could help to address this issue. \\




Our findings trace out an important dilemma for policy makers of supranational institutions, such as the European Union, when it comes to choosing a vaccination strategy. On the one hand, choosing the overall number of death cases seems to be a plausible objective. On the other hand, the supranational policy makers cannot outweigh the disutility of one country with the benefits of another country. However, our findings enhance that given the current strategy, a Pareto improvement might be possible.  


